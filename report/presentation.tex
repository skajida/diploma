\documentclass{beamer}

\usetheme{Dresden}
\usepackage{cmap}
\usepackage[T2A]{fontenc}
\usepackage[english,russian]{babel}

\usepackage{multirow}

% page count
\expandafter\def\expandafter\insertshorttitle\expandafter{\insertshorttitle\hfill\insertframenumber\,/\,\inserttotalframenumber}

\title{Минимизация обобщенных полиномов в векторном пространстве}
\author[Королёв~Ф.И.]{Королёв Фёдор Иванович}
\institute[МГУ им. М.В.~Ломоносова]{Московский государственный университет имени М.В.~Ломоносова}

\begin{document}

% title
\begin{frame}{Выпускная квалификационная работа}
\titlepage
\end{frame}

% plan
\begin{frame}{План выступления}
\tableofcontents
\end{frame}

\section[Определения]{Основные определения}
\begin{frame}{Основные определения}

\begin{block}{Булева функция}
Отображение $ f:\ E_2^n \rightarrow E_2 $ назовем \textit{булевой функцией} от $ n $ переменных.

За $ \mathbb{P}_2(n) $ обозначим множество всех булевых функций от $ n $ переменных.

Мощность этого множества равна $ \left| \mathbb{P}_2(n) \right| = 2^{2^n}$.
\end{block}

\end{frame}
\begin{frame}{Основные определения}

\begin{block}{Обобщенный полином}
\textit{Обобщенным полиномом}, реализующим булеву функцию $ f $ от $ n $ переменных, будем называть следующее представление:
$$ f(x_1, \dots, x_n) = \bigoplus\limits_{i = 1}^s K_i, $$
где $ \oplus $ --- сложение по модулю 2, $ K_i $ --- конъюнкция переменных (возможно с отрицанием) или функция $ 1 $.

За $ \mathcal{P}(n) $ обозначим множество всех обобщенных полиномов от $ n $ переменных.

Мощность этого множества равна $ \left| \mathcal{P}(n) \right| = 2^{3^n} $.
\end{block}

\end{frame}
\begin{frame}{Основные определения}

\begin{block}{Длина обобщенного полинома}
\textit{Длиной обобщенного полинома} $ p_f $, реализующего булеву функцию $ f $, назовем количество конъюнктов в его представлении.
\end{block}

\end{frame}

\section[Математическая задача]{Формулировка математической задачи}
\begin{frame}{Формулировка математической задачи}
Поскольку $ \left| \mathbb{P}(n) \right| \ne \left| \mathcal{P}(n) \right| $, то между классом булевых функций и классом обобщенных полиномов не существует взаимнооднозначного соответствия.

\begin{block}{Отношение эквивалентности}
Назовем два обобщенных полинома $ p_1,\ p_2 \in \mathcal{P}(n) $ \textit{эквивалентными}, если они реализуют одну булеву функцию $ f $.
\end{block}

Представленное отношение порождает классы эквивалентности во множестве обобщенных полиномов.

\end{frame}
\begin{frame}{Формулировка математической задачи}

\begin{block}{Математическая задача}
Поиск минимального (в смысле длины) полинома в классе эквивалентности, порожденном элементом $ p $, где $ p \in \mathcal{P}(n) $ --- входной обобщенный полином данной задачи.

Выходными данными поставленной задачи будем считать минимальный представитель класса эквивалентности, порожденного входным полиномом $ p $.
\end{block}

\end{frame}

\section{Постановка задачи}
\begin{frame}{Постановка задачи}

\begin{itemize}
    \item предложить алгоритм, реализующий эвристическую минимизацию обобщенных полиномов
    \item написать программу, реализующую предложенный алгоритм
    \item провести оценку качества ответа и быстродействия реализованной программы
    \item сравнить предложенный алгоритм с известными эвристическими алгоритмами
\end{itemize}

\end{frame}

\section{Основная часть}

\subsection{Метод прибавления нулевых полиномов}
\subsubsection{Базис нулевых полиномов}
\begin{frame}{Базис нулевых полиномов}
\textit{Базисом нулевых полиномов} $ B_n $ размерности $ n $ назовем множество всевозможных полиномов следующего вида:
$$ \overline{x_1} \land x_3 \land \dots \land ( 1 \oplus x_i \oplus \overline{x_i} ) \land \dots \land x_n. $$

Мощность базиса размерности $ n $ равна $ \left| B_n \right| = 4^n - 3^n $.

\end{frame}
\subsubsection{Алгоритм минимизации}
\begin{frame}{Алгоритм минимизации}

Из тождества алгебры логики $ a \oplus 0 \equiv a $ следует, что при прибавлении элементов базиса к исходному полиному будет сохраняться класс эквивалентности.

Тождество $ a \oplus a \equiv 0 $ позволяет <<сокращать>> одинаковые конъюнкты результирующего полинома, что позволит уменьшать длину изначального полинома после эффективного прибавления нулевого полинома.

\end{frame}
\begin{frame}{Алгоритм минимизации}

Предложен следующий эвристический алгоритм минимизации:
\begin{enumerate}
    \item Расширим базис различными сочетаниями элементов изначального базиса $ B $ до разумного ограничения по использованию оперативной памяти.
    \item Запускаем поиск сокращенного эквивалентного полинома:
    \begin{enumerate}[a.]
        \item \label{loop:begin} Рассматриваем результаты прибавления всех нулевых полиномов, имеющих с текущим хотя бы один общий конъюнкт.
        \item \label{loop:end} Выбираем тот, который позволил сократить длину полинома на наибольшее значение. При равных сокращениях выбирается полином, позже всех вошедший в базис.
        \item Алгоритм заканчивает работу, если на этапе \ref{loop:end} не нашлось ни единого кандидата, иначе возвращается к пункту \ref{loop:begin}.
    \end{enumerate}
\end{enumerate}

\end{frame}

\subsubsection{Реализация программы}
\begin{frame}{Реализация программы}
Была написана программа на языке \texttt{C++}, исходный код которой доступен по ссылке: \url{https://github.com/skajida/diploma}.
\end{frame}

\subsection{Тестирование программы}
\begin{frame}{Тестирование программы}
Программа, реализующая предложенный алгоритм, тестировалась на размерностях 3 и 4 на наборах из 100 случайно сгенерированных полиномов, длина каждого из которых бралась из распределения $ \mathcal{N} (n, 1) $ для $ n = \{ 7, 10, 13 \} $.

% Результаты представлены в таблице \ref{table_results}.

\begin{table}
\caption{Результаты работы программы}
\label{table_results}
\centering
\begin{tabular}{ |l||r|r||r|r| } \hline
\multirow{3}{*}{\bf n} & \multicolumn{4}{c|}{\bf средняя длина} \\ \cline{2-5}
    & \multicolumn{2}{c||}{3-dim} & \multicolumn{2}{c|}{4-dim} \\ \cline{2-5}
    & in    & out  & in    & out  \\ \hline \hline
7   & 6.85  & 2.21 & 7.13  & 3.89 \\ \hline
10  & 9.99  & 2.11 & 9.83  & 4.11 \\ \hline
13  & 13.01 & 2.17 & 12.96 & 4.08 \\ \hline \hline
\textbf{ср.сокращение} & \multicolumn{2}{c||}{\tt -78\%} & \multicolumn{2}{c|}{\tt -60\%} \\ \hline
\end{tabular}
\end{table}

\end{frame}

\subsection{Сравнение с известными эвристическими подходами к минимизации}

\begin{frame}{Сравнение с известными эвристическими подходами к минимизации}
Для размерностей выше 4 асимптотика роста $ C_n^k $ сочетаний элементов базиса не позволила расширить базис более, чем на всевозможные пары его элементов. По этой причине качество ответа на полиномах таких размерностей сильно уступает качеству известных эвристических алгоритмов.

Для размерностей 3 и 4 быстродействие программы значительно ниже быстродействия эвристических алгоритмов, предложенных научным сообществом.
\end{frame}

\section[Результаты]{Полученные результаты}
\begin{frame}{Полученные результаты}

\begin{itemize}
    \item был предложен алгоритм, реализующий эвристическую минимизацию обобщенных полиномов путем прибавления нулевых полиномов
    \item была написана программа, реализующая данный алгоритм
    \item была проведена оценка качества программы на случайных обобщенных полиномах размерности 3 и 4
    \item предложенный алгоритм зарекомендовал себя неконкурентоспособным в сравнении с известными эвристическими алгоритмами на размерностях пространства выше 4 с точки зрения качества ответа и на меньших размерностях с точки зрения быстродействия
\end{itemize}

\end{frame}

\section{}
\begin{frame}
\centering \huge{Спасибо за внимание!}
\end{frame}

\end{document}
