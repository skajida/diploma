\documentclass{beamer}

\usetheme{Dresden}
\usepackage{cmap}
\usepackage[T2A]{fontenc}
\usepackage[english,russian]{babel}

\usepackage{multirow}

% page count
\expandafter\def\expandafter\insertshorttitle\expandafter{\insertshorttitle\hfill\insertframenumber\,/\,\inserttotalframenumber}

\title{Минимизация обобщенных полиномов в векторном пространстве}
\author[Королёв~Ф.И.]{Королёв Фёдор Иванович}
\institute[МГУ им. М.В.~Ломоносова]{Московский государственный университет имени М.В.~Ломоносова}

\begin{document}

\begin{frame}{Выпускная квалификационная работа}
\titlepage
\end{frame}

\section[Определения]{Основные определения}
\begin{frame}{Основные определения}

\begin{block}{Булева функция}
Отображение $ f:\ E_2^n \rightarrow E_2 $ назовем \textit{булевой функцией} от $ n $ переменных.
\end{block}

\end{frame}
\begin{frame}{Основные определения}

\begin{block}{Обобщенный полином}
\textit{Обобщенным полиномом}, реализующим булеву функцию $ f $ от $ n $ переменных, будем называть следующее представление:
$$ f(x_1, \dots, x_n) = \bigoplus\limits_{i = 1}^s K_i, $$
где $ \oplus $ --- сложение по модулю 2, $ K_i $ --- конъюнкция переменных (возможно с отрицанием) или функция $ 1 $.
\end{block}

\end{frame}
\begin{frame}{Основные определения}

\begin{block}{Длина обобщенного полинома}
\textit{Длиной обобщенного полинома} $ p_f $, реализующего булеву функцию $ f $, назовем количество конъюнктов в его представлении.
\end{block}

\end{frame}

\section[Математическая задача]{Формулировка математической задачи}
\begin{frame}{Формулировка математической задачи}

\begin{block}{Математическая задача}
Поиск минимального (в смысле длины) обобщенного полинома, реализующего ту же функцию, что и входной обобщенный полином. Выходными данными поставленной задачи будем считать найденный обобщенный полином.
\end{block}

\end{frame}

\section{Постановка задачи}
\begin{frame}{Постановка задачи}

\begin{itemize}
    \item предложить эвристический подход к решению задачи минимизации обобщенных полиномов
    \item написать программу, реализующую предложенный алгоритм
    \item привести оценку качества ответа и быстродействия реализованной программы
    \item сравнить предложенный алгоритм с известными алгоритмами минимизации обобщенных полиномов
\end{itemize}

\end{frame}

\section{Основная часть}

\begin{frame}{Алгоритм минимизации}

\begin{enumerate}
    \item Генерируем базис (множество) нулевых обобщенных полиномов, расширяем его сочетаниями его элементов до разумного ограничения по использованию оперативной памяти.
    \item Запускаем поиск сокращенного эквивалентного полинома:
    \begin{enumerate}[a.]
        \item \label{loop:begin} Рассматриваем результаты прибавления всех нулевых полиномов, имеющих с текущим хотя бы один общий конъюнкт.
        \item \label{loop:end} Выбираем тот, который позволил сократить длину полинома на наибольшее значение. При равных сокращениях выбирается полином, позже всех вошедший в базис.
        \item Алгоритм заканчивает работу, если на этапе \ref{loop:end} не нашлось ни единого кандидата, иначе возвращается к пункту \ref{loop:begin}.
    \end{enumerate}
\end{enumerate}

\end{frame}

\begin{frame}{Реализация программы}
Была написана программа на языке \texttt{C++}, исходный код которой доступен по ссылке: \url{https://github.com/skajida/diploma}.
\end{frame}

\begin{frame}{Тестирование программы}
Программа тестировалась в пространстве обобщенных полиномов размерности 3 и 4 на наборах из 100 случайно сгенерированных полиномов, длина каждого из которых бралась из распределения $ \mathcal{N} (n, 1) $ для $ n = \{ 7, 10, 13 \} $.

\begin{table}
\caption{Результаты работы программы}
\label{table_results}
\centering
\begin{tabular}{ |l||r|r||r|r| } \hline
\multirow{3}{*}{\bf n} & \multicolumn{4}{c|}{\bf средняя длина} \\ \cline{2-5}
    & \multicolumn{2}{c||}{3-dim} & \multicolumn{2}{c|}{4-dim} \\ \cline{2-5}
    & in    & out  & in    & out  \\ \hline \hline
7   & 6.85  & 2.21 & 7.13  & 3.89 \\ \hline
10  & 9.99  & 2.11 & 9.83  & 4.11 \\ \hline
13  & 13.01 & 2.17 & 12.96 & 4.08 \\ \hline \hline
\textbf{ср.сокращение} & \multicolumn{2}{c||}{\tt -78\%} & \multicolumn{2}{c|}{\tt -60\%} \\ \hline
\end{tabular}
\end{table}

\end{frame}

\begin{frame}{Сравнение с известными эвристическими подходами}
Для размерностей выше 4 асимптотика роста $ C_n^k $ сочетаний элементов базиса не позволила расширить базис более, чем на всевозможные пары его элементов. По этой причине качество ответа на полиномах таких размерностей сильно уступает качеству известных эвристических подходов.

Для размерностей 3 и 4 пространства обобщенных полиномов быстродействие программы значительно ниже быстродействия эвристических подходов, предложенных научным сообществом.
\end{frame}

\section[Результаты]{Полученные результаты}
\begin{frame}{Полученные результаты}

\begin{itemize}
    \item был предложен эвристический подход к минимизации обобщенных полиномов путем прибавления нулевых полиномов
    \item была написана программа, реализующая предложенный алгоритм
    \item была проведена оценка качества программы на случайных обобщенных полиномах из пространств обобщенных полиномов размерностей 3 и 4
\end{itemize}

\end{frame}

\section{}
\begin{frame}
\centering \huge{Спасибо за внимание!}
\end{frame}

\end{document}
