\documentclass[a4paper,12pt,titlepage]{article}

\usepackage{cmap}
\usepackage[T2A]{fontenc}
\usepackage[english,russian]{babel}
\usepackage{geometry,indentfirst,graphicx}
\usepackage[hidelinks]{hyperref}

\usepackage{amssymb,amsmath}
\usepackage{multirow}

\setcounter{secnumdepth}{0}
\linespread{1.5}
\geometry{a4paper,left=30mm,top=20mm,bottom=20mm,right=15mm}

\begin{document}

\hypersetup{pageanchor=false}
\begin{titlepage}
    \begin{center}
        \includegraphics{logo.pdf} \\
        \textsc{\small Московский государственный университет имени М.~В.~Ломоносова \\
        Факультет вычислительной математики и кибернетики \\
        Кафедра математической кибернетики \\}
        \vfill
        \large{Королёв Фёдор Иванович} \\
        \vspace{1cm}
        \textbf{\large Минимизация обобщенных полиномов в векторном пространстве} \\
        \vfill
        \textsc{\large Выпускная квалификационная работа} \\
    \end{center}
    \begin{flushright}
        \vfill
        \textbf{Научный руководитель:} \\
        к.ф.-м.н. Бухман~А.~В.
    \end{flushright}
    \begin{center}
        \vfill
        {\small Москва \\
        2022}
    \end{center}
\end{titlepage}

\hypersetup{pageanchor=true}
\tableofcontents
\newpage

\section{Основные понятия}

Пусть $ E_2 $ --- множество $ \{ 0, 1 \} $.\\
Отображение $ f:\ E_2^n \rightarrow E_2 $ назовем булевой функцией от $ n $ переменных.\\
$ \mathbb{P}_2(n) $ --- множество всех булевых функций от $ n $ переменных.\\
Если $ \sigma \in E_2 $, то введем обозначение
\begin{equation*}
    x^\sigma =
    \begin{cases}
        x & \text{при $ \sigma = 1 $}\\
        \overline x & \text{при $ \sigma = 0 $,}
    \end{cases}
\end{equation*}
выражение $ x^\sigma $ будем называть \textit{литералом}.\\
\textit{Элементарная конъюнкция} ранга $ r $ --- это выражение вида $ x_{i_1}^{\sigma_1} \dots x_{i_r}^{\sigma_r} $, где $ x_{i_1}, \dots, x_{i_r} $ --- различные переменные и $ \sigma_1, \dots, \sigma_r \in E_2 $. Элементарной конъюнкцией ранга 0 назовем константу 1.\\
\textit{Обобщенный полином} --- выражение вида $ \bigoplus\limits_{i = 1}^s K_i $, где $ K_i $ --- элементарные конъюнкции.\\
\textit{Длиной обобщенного полинома} назовем количество элементарных конъюнкций в нем.\\
$ \mathcal{P}(n) $ --- множество всех обобщенных полиномов от $ n $ переменных.

\section{Введение}

\subsection{Задача минимизации обобщенных полиномов}

Одним из возможных способов задать булеву функцию $ f $ от $ n $ переменных является задание ее таблицы истинности, количество строк которой равно $ 2^n $. Отсюда следует, что мощность класса булевых функций от $ n $ переменных $ \left| \mathbb{P}_2(n) \right| $ равна $ 2^{2^n} $, поскольку на каждую строку таблицы истинности булева функция может ответить нулем или единицей, то есть элементами из множества $ \{ 0,\ 1 \} $ мощности 2.

Элементы класса обобщенных полиномов $ \mathcal{P}(n) $ от $ n $ переменных задаются, в соответствии с представленным определением, исключающим или, примененным ко всем вхождениям в полином элементарных конъюнкций. Мощность класса элементарных конъюнкций равна $ 3^n $, поскольку $ n $ переменных могут принять одно из значений множества $ \{ 1, x_i, \overline x_i \},\ i = \overline{1,\ n} $. Тогда каждый конъюнкт из класса элементарных конъюнкций может входить или не входить в обобщенный полином, поэтому мощность множества всех обобщенных полиномов $ n $ переменных $ \left| \mathcal{P}(n) \right| $ равна $ 2^{3^n} $.

Поскольку $ \left| \mathbb{P}(n) \right| \ne \left| \mathcal{P}(n) \right| $, то между классом булевых функций и классом обобщенных полиномов не существует взаимнооднозначного соответствия. В таком случае введем отношение эквивалентности в классе обобщенных полиномов. Назовем два обобщенных полинома $ p_1,\ p_2 \in \mathcal{P}(n) $ эквивалентными, если они реализуют одну булеву функцию.

Это отношение порождает классы эквивалентности во множестве обобщенных полиномов. Сформулируем математическую задачу: поиск минимального (в смысле длины) полинома в классе эквивалентности, порожденном элементом $ p $, где $ p $ --- входной обобщенный полином данной задачи. Его представлением будем считать набор элементарных конъюнкций, входящих в него. Выходными данными поставленной задачи будем считать минимальный представитель класса эквивалентности, порожденного входным полиномом $ p $.

Рассмотрим некоторые характеристики представления булевых функций обобщенными полиномами.\\
Для $ P \in \mathcal{P}(n) $ обозначим длину полинома $ l(P) $. Введем функционал
$$ l(f) = \min\limits_{P_f \in \mathcal{P}_f} l(P_f), $$
обозначающий минимальную длину обобщенного полинома среди всех, реализующих булеву функцию $ f $. Функцией Шеннона сложности в классе обобщенных полиномов называется функция
$$ L_\text{о.п.} = \max\limits_{f \in \mathbb{P}_2(n)} l(f). $$

Существуют верхняя и нижняя оценки для функции Шеннона в классе обобщенных полиномов, представим их \cite{selezn}:
$$ \frac{2^n}{n \log_2 3} \le L_\text{о.п.}(n) \le 2 \cdot \frac{2^n}{n} (1 + \ln n) $$

\subsection{Развитие исследований}

Интерес задачи минимизации обобщенных полиномов представляют преимущества, которыми они обладают в сравнении с дизъюнктивными нормальными формами. В производстве цифровых схем на базе логических вентилей \textsc{AND} и \textsc{XOR} уменьшены площадь схемы и время задержки \cite{delay} в сравнении со схемами на базе \textsc{AND} и \textsc{OR}. Текущее применение \textsc{AND-XOR} выражений заключается в синтезе таких двухуровневых схем, как обратимые логические\footnote{схемы, использующие вентили с характерным низким энергопотреблением, реализующие обратимые вычисления \cite{revsynth}} \cite{reversible} и квантовые \cite{quantum}.

Некоторые исследователи предлагают алгоритмы, находящие точный полиномиальный минимум \cite{min-tau,exact6,exact} функции, однако до сих пор не был найден эффективный метод решения таким образом поставленной задачи для булевых функций более восьми переменных в общем случае \cite{exact8}, поэтому особый интерес представляет эвристический подход к минимизации обобщенных полиномов.

Научным сообществом было предложено множество эвристических алгоритмов минимизации, не гарантирующих минимальности полученного в качестве результата полинома, однако позволяющих добиться высокого быстродействия в сравнении с подходами с поиском точного минимума. Примерами могут служить алгоритмы Сасао \cite{exmin2}, Стергиу-Вудурис-Папаконстантину \cite{mvesopmin}, Мищенко-Перковски \cite{exorcism4}. Стоит упомянуть, что некоторые алгоритмы (например \textsc{EXMIN2} \cite{exmin2}) принимают на вход дизъюнктивную нормальную форму, но на первом алгоритмическом этапе занимаются приведением ее к обобщенному виду.

Существуют и алгоритмы минимизации \cite{exorcism-mv3}, удовлетворяющие условию крайне высокого быстродействия. Такой подход находит применение в алгоритмах, использующих минимизацию обобщенных полиномов в цикле другой минимизирующей программы, например в машинном обучении \cite{machine-learning}.

\section{Постановка задачи}

Зафиксируем следующие задачи:
\begin{enumerate}
    \item предложить алгоритм, реализующий эвристическую минимизацию обобщенных полиномов
    \item написать программу, реализующую предложенный алгоритм
    \item провести оценку качества ответа и быстродействия реализованной программы
    \item сравнить предложенный алгоритм с известными эвристическими алгоритмами
\end{enumerate}

\section{Основная часть}

\subsection{Метод прибавления нулевых полиномов}

Задан исходный обобщенный полином в виде $ \bigoplus\limits_i K_i $. Генерируется базис нулевых полиномов --- обобщенных полиномов, тождественно равных нулю. Поскольку $ a \oplus 0 \equiv a $, то операция прибавления элементов базиса нулевых полиномов к исходному полиному будет сохранять класс эквивалентности, при этом может повлечь за собой сокращение длины полинома (поскольку для некоторых конъюнктов может сработать правило сокращения $ a \oplus a \equiv 0 $).

\subsubsection{Описание базиса нулевых полиномов}
Под базисом нулевых полиномов здесь и далее будем понимать множество нулевых полиномов, сгенерированных следующим образом.
Представим себе элементарную конъюнкцию из $ n $ множителей. На позиции i-того литерала выставим один из четырех элементов множества $ \{ 0, 1, x_i, \overline{x_i} \} $. Очевидно, что конъюнкция тождественно равна 0 тогда и только тогда, когда на месте хотя бы одного литерала выставлен 0. Пусть на места всех $ n $ литералов выставлены элементы описанного множества, при этом хотя бы один из них равен $ 0 \equiv 1 \oplus x_i \oplus \overline x_i $, тогда в силу дистрибутивности конъюнкции относительно исключающего или можно раскрыть скобки и получить выражение, являющееся, в силу тождественного равенства нулю, нулевым обобщенным полиномом. Весь базис нулевых полиномов, построенный на множестве $ \{ 0, 1, x_i, \overline{x_i} \} $, будет состоять из $ 4^n $ (всех вариантов расстановки элементов множества на $ n $ позиций) минус $ 3^n $ (всех вариантов расстановки элементов $ \{ 1, x_i, \overline{x_i} \} $ на $ n $ позиций, поскольку выставив исключительно элементы данного подмножества на места литералов выражение не будет удовлетворять необходимому условию тождественного равенства нулю). Итого, мощность описанного базиса $ \left| B \right| = 4^n - 3^n. $

\subsubsection{Алгоритм минимизации}

В настоящей работе предложен следующий эвристический алгоритм минимизации обобщенных полиномов:
\begin{enumerate}
    \item Генерируем описанный выше базис нулевых обобщенных полиномов.
    \item Расширим базис $ C_{\left| B \right|}^2 $ комбинациями всевозможных пар изначального базиса $ B $, $ C_{\left| B \right|}^3 $ тройками по аналогии, и так далее до разумного ограничения по использованию оперативной памяти.
    \item Запускаем поиск сокращенного эквивалентного полинома:
    \begin{enumerate}
        \item \label{loop:begin} Рассматриваем результаты прибавления всех нулевых полиномов из расширенного сочетаниями базиса, имеющих с текущим хотя бы один общий конъюнкт с текущим представлением сокращаемого полинома.
        \item \label{loop:end} Выбираем из всех вариантов тот, который позволил сократить длину полинома на наибольшее значение. В случае одинаковых сокращений выбирается сокращение с последним вошедшим в базис нулевым полиномом.
        \item Алгоритм заканчивает работу, если на этапе \ref{loop:end} не нашлось ни единого кандидата, иначе возвращается к пункту \ref{loop:begin}.
    \end{enumerate}
\end{enumerate}

Алгоритм возвращает сокращенный обобщенный полином в виде $ \bigoplus\limits_i K_i $.

Настоящий алгоритм отличается от его предыдущей версии тем, что за счет использования сочетаний элементов изначального базиса удалось значительно улучшить скорость работы программы, что позволило пробовать использовать более <<глубокие>> последовательности прибавлений (ранее прибавлялись только сами элементы базиса и их всевозможные пары).

\subsubsection{Реализация программы}

Была написана программа на языке \texttt{C++}, исходный код которой доступен по ссылке: \url{https://github.com/skajida/diploma}.

\subsection{Тестирование}

Как упоминалось выше, основным фактором, мешающим дополнять базис нулевых полиномов комбинациями элементов изначально сгенерированного базиса (пусть далее его мощность $ n $), является ограниченность оперативной памяти. При очередной попытке расширить базис $ C_n^k $ комбинациями из $ k $ элементов изначального базиса, программа оценивает ожидаемый запрос памяти и, если он превышает 8 GB, то отклоняет попытку расширения.

Таким образом удалось расширить базис в пространстве размерности 3 до комбинаций из 5 элементов изначального базиса, а в пространстве размерности 4 до комбинаций из 3 элементов. Для размерности 5 по оценке потребовалось бы порядка 80 GB оперативной памяти для хранения самого базиса, комбинаций из 2 и 3 его элементов, поэтому для размерностей выше 4 проэкспериментировать с расширением базиса до всевозможных троек не удалось.

По этой причине в настоящей работе будут обсуждаться результаты работы разных подходов именно для размерностей 3 и 4.

В статье Папаконстантину \cite{513100} был озвучен следующий тезис: <<была написана программа, которой на вход подавались 100 случайно сгенерированных функций 5-ти переменных со средним количеством конъюнктов равным 13>>, <<средняя найденная алгоритмом длина равна 5.9, что лучше, чем результат около 6.4 у \cite{beat1, beat2}>>.

Так как для размерности 5 настоящему алгоритму не удается расширить базис до троек, то используем данный подход к тестированию работы программы для размерностей 3 и 4 с расширением сочетаниями до пятерок и троек соответственно. В таблице \ref{table_results} отображены длины входных и выходных данных программы в тестах со 100 случайно сгенерированными полиномами средней длины 7, 10, и 13 соответственно. Длина отдельно взятого полинома из входных данных выбиралась из нормального распределения с математическим ожиданием, равным желаемой средней длине, и дисперсией, равной 1. Конъюнкты подбирались из равномерного распределения по их лексикографическому индексу до тех пор, пока длина полинома не будет равной заявленной.

\begin{table}[h!]
\centering
\begin{tabular}{ |l||r|r||r|r||r|r||r|r||r|r||r|r| } \hline
\multirow{4}{*}{\bf длина} & \multicolumn{12}{c|}{\bf количество полиномов} \\ \cline{2-13}
    & \multicolumn{2}{c||}{3-dim} & \multicolumn{2}{c||}{4-dim} & \multicolumn{2}{c||}{3-dim} & \multicolumn{2}{c||}{4-dim} & \multicolumn{2}{c||}{3-dim} & \multicolumn{2}{c|}{4-dim} \\ \cline{2-13}
    & \texttt{in} & \texttt{out} & \texttt{in} & \texttt{out} & \texttt{in} & \texttt{out} & \texttt{in} & \texttt{out} & \texttt{in} & \texttt{out} & \texttt{in} & \texttt{out} \\ \hline \hline
1   & 0     & 9     & 0     & 0     & 0     & 7     & 0     & 0     & 0     & 10    & 0     & 0     \\ \hline
2   & 0     & 61    & 0     & 4     & 0     & 75    & 0     & 4     & 0     & 63    & 0     & 4     \\ \hline
3   & 0     & 30    & 0     & 31    & 0     & 18    & 0     & 23    & 0     & 27    & 0     & 24    \\ \hline
4   & 3     & 0     & 0     & 40    & 0     & 0     & 0     & 41    & 0     & 0     & 0     & 42    \\ \hline
5   & 5     & 0     & 3     & 22    & 0     & 0     & 0     & 22    & 0     & 0     & 0     & 21    \\ \hline
6   & 26    & 0     & 22    & 3     & 0     & 0     & 0     & 10    & 0     & 0     & 0     & 8     \\ \hline
7   & 43    & 0     & 39    & 0     & 2     & 0     & 0     & 0     & 0     & 0     & 0     & 1     \\ \hline
8   & 17    & 0     & 31    & 0     & 4     & 0     & 9     & 0     & 0     & 0     & 0     & 0     \\ \hline
9   & 5     & 0     & 5     & 0     & 23    & 0     & 31    & 0     & 0     & 0     & 0     & 0     \\ \hline
10  & 1     & 0     & 0     & 0     & 41    & 0     & 35    & 0     & 1     & 0     & 2     & 0     \\ \hline
11  & 0     & 0     & 0     & 0     & 25    & 0     & 18    & 0     & 4     & 0     & 4     & 0     \\ \hline
12  & 0     & 0     & 0     & 0     & 4     & 0     & 7     & 0     & 24    & 0     & 28    & 0     \\ \hline
13  & 0     & 0     & 0     & 0     & 1     & 0     & 0     & 0     & 38    & 0     & 33    & 0     \\ \hline
14  & 0     & 0     & 0     & 0     & 0     & 0     & 0     & 0     & 30    & 0     & 28    & 0     \\ \hline
15  & 0     & 0     & 0     & 0     & 0     & 0     & 0     & 0     & 3     & 0     & 5     & 0     \\ \hline \hline
\textbf{mean len}   & 6.85  & 2.21  & 7.13  & 3.89  & 9.99  & 2.11  & 9.83  & 4.11  & 13.01 & 2.17  & 12.96 & 4.08  \\ \hline
\textbf{reduct}  & \multicolumn{2}{c||}{\tt -68\%}  & \multicolumn{2}{c||}{\tt -45\%} & \multicolumn{2}{c||}{\tt -79\%}  & \multicolumn{2}{c||}{\tt -58\%}   & \multicolumn{2}{c||}{\tt -83\%}  & \multicolumn{2}{c|}{\tt -69\%} \\ \hline
\textbf{time}       & \multicolumn{2}{c||}{\tt 4m 44s}  & \multicolumn{2}{c||}{\tt 15m 47s} & \multicolumn{2}{c||}{\tt 8m 25s}  & \multicolumn{2}{c||}{\tt 24m 09s}  & \multicolumn{2}{c||}{\tt 8m 43s}  & \multicolumn{2}{c|}{\tt 30m 38s}  \\ \hline
\end{tabular}
\caption{Результаты работы программы на полиномах длины 7, 10, 13}
\label{table_results}
\end{table}

\subsection{Сравнение с предыдущей реализацией}

Предыдущая версия программы работала только с изначальным базисом и его всевозможными парами, качественное сравнение результатов представлено в таблице \ref{table_comparison}, где под <<глубиной>> (depth) подразумевается максимальное $ k $ рассматриваемых $ C_n^k $ комбинаций элементов базиса.

\begin{table}[h!]
\centering
\begin{tabular}{ |l||r|r||r|r|} \hline
\multirow{3}{*}{\bf длина входа} & \multicolumn{4}{c|}{\bf средняя длина выхода/сокращение} \\ \cline{2-5}
    & \multicolumn{2}{c||}{3-dim} & \multicolumn{2}{c|}{4-dim} \\ \cline{2-5}
    & \texttt{2-depth} & \texttt{5-depth} & \texttt{2-depth} & \texttt{3-depth} \\ \hline \hline
7   & 2.95/-57\%  & 2.21/-68\%  & 4.98/-30\%  & 3.89/-45\%  \\ \hline
10  & 2.93/-71\%  & 2.11/-79\%  & 6.11/-38\%  & 4.11/-58\%  \\ \hline
13  & 2.97/-77\%  & 2.17/-83\%  & 6.24/-52\%  & 4.08/-69\%  \\ \hline \hline
\textbf{ср.преимущество}    & \multicolumn{2}{c||}{12\%}  & \multicolumn{2}{c|}{43\%}   \\ \hline
\end{tabular}
\caption{Сравнение с малой глубиной}
\label{table_comparison}
\end{table}

\subsection{Сравнение с опубликованными эвристическими алгоритмами}

Поскольку для размерности выше 4-х алгоритм не использует глубину больше двух, то с точки зрения качества ответа он не претендует на конкуренцию с известными эвристическими алгоритмами минимизации \cite{exmin2, exorcism4, exorcism-mv3, grmin2}.

Также в 1992 году была опубликована работа Сасао \cite{4var-min}, в которой описаны свойства минимальных полиномов размерности 4, откуда следует, что все булевы функции представимы обобщенными полиномами длины строго меньше 7. Однако, в таблице \ref{table_results} видно, что из 300 случайно сгенерированных полиномов один из них был приведен к полиному длины 7, что доказывает, что предложенный алгоритм не находит точных полиномиальных минимумов.

С точки зрения производительности алгоритм также оставляет желать лучшего: несмотря на то, что с предыдущей версии была проведена значительная работа по увеличению скорости работы, что позволило рассматривать более глубокие вариации базиса, алгоритм на полиномах размерности 4 тратит в среднем по 14.1 секунд (исходя из временных замеров, представленных в таблице \ref{table_results}), что не идет в сравнение с алгоритмами \cite{exmin2, exorcism4, grmin2}, тем более со специально разработанным для встраивания в другие алгоритмы методом \textsc{EXORCISM-MV3} \cite{exorcism-mv3}.

Данные замечания приводят к выводу о том, что предложенный метод нуждается в значительной доработке, чтобы конкурировать с опубликованными эвристическими алгоритмами минимизации обобщенных полиномов.

\section{Полученные результаты}

\begin{enumerate}
    \item был предложен алгоритм, реализующий эвристическую минимизацию обобщенных полиномов путем прибавления нулевых полиномов
    \item была написана программа, реализующая данный алгоритм
    \item была проведена оценка качества программы на случайных обобщенных полиномах размерности 3 и 4
\end{enumerate}

\begin{raggedright}
    \addcontentsline{toc}{section}{Литература}
    \begin{thebibliography}{99}
        \bibitem{delay} Kalay~U., Hall~D.~V., Perkowski~M.~A. A minimal universal test set for self-test of EXOR-sum-of-products circuits //IEEE Transactions on Computers. – 2000. – Т.~49. – №.~3. – С.~267-276.
        \bibitem{reversible} Yang~G. et al. Majority-based reversible logic gates //Theoretical computer science. – 2005. – Т.~334. – №.~1-3. – С.~259-274.
        \bibitem{quantum} Iwama~K., Kambayashi~Y., Yamashita~S. Transformation rules for designing CNOT-based quantum circuits //Proceedings of the 39th annual Design Automation Conference. – 2002. – С.~419-424.
        \bibitem{revsynth} Shende~V.~V. et al. Reversible logic circuit synthesis //Proceedings of the 2002 IEEE/ACM international conference on Computer-aided design. – 2002. – С.~353-360.
        \bibitem{min-tau} Hirayama~T., Nishitani~Y., Sato~T. A faster algorithm of minimizing AND-EXOR expressions //IEICE transactions on fundamentals of electronics, communications and computer sciences. – 2002. – Т.~85. – №.~12. – С.~2708-2714.
        \bibitem{exact6} Gaidukov~A. Algorihm to derive minimum ESOP for 6-variable function //5th International Workshop on Boolean Problems, Sept.~2002. – 2002.
        \bibitem{exact} Sasao~T. An exact minimization of AND-EXOR expressions using reduced covering functions //Proc. of the Synthesis and Simulation Meeting and International Interchange. – 1993. – С.~374-383.
        \bibitem{exact8} Stergiou~S., Papakonstantinou~G. Exact minimization of ESOP expressions with less than eight product terms //Journal of Circuits, Systems, and Computers. – 2004. – Т.~13. – №.~01. – С.~1-15.
        \bibitem{exmin2} Sasao~T. EXMIN2: A simplification algorithm for exclusive-OR-sum-of-products expressions for multiple-valued-input two-valued-output functions //IEEE Transactions on Computer-Aided Design of Integrated Circuits and Systems. – 1993. – Т.~12. – №.~5. – С.~621-632.
        \bibitem{mvesopmin} Stergiou~S., Voudouris~D., Papakonstantinou G. Multiple-value exclusive-or sum-of-products minimization algorithms //IEICE TRANSACTIONS on Fundamentals of Electronics, Communications and Computer Sciences. – 2004. – Т.~87. – №.~5. – С.~1226-1234.
        \bibitem{exorcism4} Mishchenko~A., Perkowski~M. Fast heuristic minimization of exclusive-sums-of-products. – 2001.
        \bibitem{exorcism-mv3} Song~N., Perkowski~M. New fast approach to approximate ESOP minimization for incompletely specified multi-output functions //Proc. IFIP 10.5 Workshop on Application of Reed-Muller Expansions in Circuit Design. – 1997. – С. 61-72.
        \bibitem{machine-learning} Perkowski~M. et al. Application of ESOP minimization in machine learning and knowledge discovery //Proc. Reed Muller. – 1995. – Т. 95.
        \bibitem{selezn} Селезнева~С.~Н., Шуплецов~М.~С., Дайняк~А.~Б. Булевы функции и полиномы //Москва – 2006.
        \bibitem{513100} Papakonstantinou~G. Minimization of modulo-2 sum of products //IEEE Transactions on Computers. – 1979. – Т. 28. – №. 02. – С. 163-167.
        \bibitem{beat1} Even~S., Kohavi~I., Paz~A. On minimal modulo 2 sums of products for switching functions //IEEE Transactions on Electronic Computers. – 1967. – №. 5. – С. 671-674.
        \bibitem{beat2} Bioul~G. Minimization of ring-sum expansions of Boolean functions //Philips Res. Repts. – 1973. – Т. 28. – С. 17-36.
        \bibitem{grmin2} Debnath~D., Sasao~T. GRMIN2: A heuristic simplification algorithm for generalised Reed-Muller expressions //IEE Proceedings-Computers and Digital Techniques. – 1996. – Т.~143. – №.~6. – С.~376-384.
        \bibitem{4var-min} Koda~N., Sasao~T. Four‐variable AND‐EXOR minimum expressions and their properties //Systems and computers in Japan. – 1992. – Т. 23. – №. 10. – С. 27-41.
    \end{thebibliography}
\end{raggedright}

\end{document}
