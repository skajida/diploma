\documentclass[a4paper,12pt,titlepage]{article}

\usepackage{cmap}
\usepackage[T2A]{fontenc}
\usepackage[english,russian]{babel}
\usepackage{geometry}
\usepackage{hyperref}
\usepackage{graphicx}

\setcounter{secnumdepth}{0}
\geometry{a4paper,left=30mm,top=30mm,bottom=30mm,right=30mm}
\hypersetup{
    citebordercolor=1 1 1,
    linkbordercolor=1 1 1,
    urlbordercolor=1 1 1
}

\begin{document}

\hypersetup{pageanchor=false}
\begin{titlepage}
    \begin{center}
        \includegraphics{logo.pdf} \\
        \textsc{\small Московский государственный университет имени М.~В.~Ломоносова \\
        Факультет вычислительной математики и кибернетики \\
        Кафедра математической кибернетики \\}
        \vfill
        \large{Королёв Фёдор Иванович} \\
        \vspace{1cm}
        \textbf{\large Минимизация обобщенных полиномов в векторном пространстве} \\
        \vfill
        \textsc{\large Выпускная квалификационная работа} \\
    \end{center}
    \begin{flushright}
        \vfill
        \textbf{Научный руководитель:} \\
        к.ф.-м.н. Бухман~А.~В.
    \end{flushright}
    \begin{center}
        \vfill
        {\small Москва\\2022}
    \end{center}
\end{titlepage}

\hypersetup{pageanchor=true}
\tableofcontents
\newpage

\section{Введение}

В этом разделе определяется актуальность темы работы, формулируется ее цель и задачи, даются историческая справка о возникновении и развитии исследований этой задачи, известные результаты.

\section{Основные понятия}

В этом разделе вводятся основные понятия и описываются известные результаты, касающиеся рассматриваемой задачи.

\section{Постановка задачи}

В этом разделе четко формулируется, что требовалось сделать в работе: исследовать такие-то свойства, доказать такие-то теоремы, реализовать такие-то программы и т.д.

\section{Основная часть}

В этом разделе подробно описываются полученные результаты.

\section{Полученные результаты}

В этом разделе сжато перечисляются полученные результаты: доказаны такие-то теоремы, реализованы такие-то программы, получены такие-то экспериментальные результаты и т.д.

\begin{raggedright}
\addcontentsline{toc}{section}{Литература}
\begin{thebibliography}{99}
    \bibitem{selezn} Селезнева~С.~Н., Шуплецов~М.~С., Дайняк~А.~Б. Булевы функции и полиномы //Москва – 2006. – С.~9-13.
\end{thebibliography}
В этом разделе приводится список литературы, использованной при подготовке ВКР. На каждый пункт списка должна иметься ссылка в тексте ВКР.
\end{raggedright}

\end{document}
